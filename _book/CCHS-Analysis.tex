% Options for packages loaded elsewhere
\PassOptionsToPackage{unicode}{hyperref}
\PassOptionsToPackage{hyphens}{url}
%
\documentclass[
]{book}
\usepackage{amsmath,amssymb}
\usepackage{iftex}
\ifPDFTeX
  \usepackage[T1]{fontenc}
  \usepackage[utf8]{inputenc}
  \usepackage{textcomp} % provide euro and other symbols
\else % if luatex or xetex
  \usepackage{unicode-math} % this also loads fontspec
  \defaultfontfeatures{Scale=MatchLowercase}
  \defaultfontfeatures[\rmfamily]{Ligatures=TeX,Scale=1}
\fi
\usepackage{lmodern}
\ifPDFTeX\else
  % xetex/luatex font selection
\fi
% Use upquote if available, for straight quotes in verbatim environments
\IfFileExists{upquote.sty}{\usepackage{upquote}}{}
\IfFileExists{microtype.sty}{% use microtype if available
  \usepackage[]{microtype}
  \UseMicrotypeSet[protrusion]{basicmath} % disable protrusion for tt fonts
}{}
\makeatletter
\@ifundefined{KOMAClassName}{% if non-KOMA class
  \IfFileExists{parskip.sty}{%
    \usepackage{parskip}
  }{% else
    \setlength{\parindent}{0pt}
    \setlength{\parskip}{6pt plus 2pt minus 1pt}}
}{% if KOMA class
  \KOMAoptions{parskip=half}}
\makeatother
\usepackage{xcolor}
\usepackage{color}
\usepackage{fancyvrb}
\newcommand{\VerbBar}{|}
\newcommand{\VERB}{\Verb[commandchars=\\\{\}]}
\DefineVerbatimEnvironment{Highlighting}{Verbatim}{commandchars=\\\{\}}
% Add ',fontsize=\small' for more characters per line
\usepackage{framed}
\definecolor{shadecolor}{RGB}{248,248,248}
\newenvironment{Shaded}{\begin{snugshade}}{\end{snugshade}}
\newcommand{\AlertTok}[1]{\textcolor[rgb]{0.94,0.16,0.16}{#1}}
\newcommand{\AnnotationTok}[1]{\textcolor[rgb]{0.56,0.35,0.01}{\textbf{\textit{#1}}}}
\newcommand{\AttributeTok}[1]{\textcolor[rgb]{0.13,0.29,0.53}{#1}}
\newcommand{\BaseNTok}[1]{\textcolor[rgb]{0.00,0.00,0.81}{#1}}
\newcommand{\BuiltInTok}[1]{#1}
\newcommand{\CharTok}[1]{\textcolor[rgb]{0.31,0.60,0.02}{#1}}
\newcommand{\CommentTok}[1]{\textcolor[rgb]{0.56,0.35,0.01}{\textit{#1}}}
\newcommand{\CommentVarTok}[1]{\textcolor[rgb]{0.56,0.35,0.01}{\textbf{\textit{#1}}}}
\newcommand{\ConstantTok}[1]{\textcolor[rgb]{0.56,0.35,0.01}{#1}}
\newcommand{\ControlFlowTok}[1]{\textcolor[rgb]{0.13,0.29,0.53}{\textbf{#1}}}
\newcommand{\DataTypeTok}[1]{\textcolor[rgb]{0.13,0.29,0.53}{#1}}
\newcommand{\DecValTok}[1]{\textcolor[rgb]{0.00,0.00,0.81}{#1}}
\newcommand{\DocumentationTok}[1]{\textcolor[rgb]{0.56,0.35,0.01}{\textbf{\textit{#1}}}}
\newcommand{\ErrorTok}[1]{\textcolor[rgb]{0.64,0.00,0.00}{\textbf{#1}}}
\newcommand{\ExtensionTok}[1]{#1}
\newcommand{\FloatTok}[1]{\textcolor[rgb]{0.00,0.00,0.81}{#1}}
\newcommand{\FunctionTok}[1]{\textcolor[rgb]{0.13,0.29,0.53}{\textbf{#1}}}
\newcommand{\ImportTok}[1]{#1}
\newcommand{\InformationTok}[1]{\textcolor[rgb]{0.56,0.35,0.01}{\textbf{\textit{#1}}}}
\newcommand{\KeywordTok}[1]{\textcolor[rgb]{0.13,0.29,0.53}{\textbf{#1}}}
\newcommand{\NormalTok}[1]{#1}
\newcommand{\OperatorTok}[1]{\textcolor[rgb]{0.81,0.36,0.00}{\textbf{#1}}}
\newcommand{\OtherTok}[1]{\textcolor[rgb]{0.56,0.35,0.01}{#1}}
\newcommand{\PreprocessorTok}[1]{\textcolor[rgb]{0.56,0.35,0.01}{\textit{#1}}}
\newcommand{\RegionMarkerTok}[1]{#1}
\newcommand{\SpecialCharTok}[1]{\textcolor[rgb]{0.81,0.36,0.00}{\textbf{#1}}}
\newcommand{\SpecialStringTok}[1]{\textcolor[rgb]{0.31,0.60,0.02}{#1}}
\newcommand{\StringTok}[1]{\textcolor[rgb]{0.31,0.60,0.02}{#1}}
\newcommand{\VariableTok}[1]{\textcolor[rgb]{0.00,0.00,0.00}{#1}}
\newcommand{\VerbatimStringTok}[1]{\textcolor[rgb]{0.31,0.60,0.02}{#1}}
\newcommand{\WarningTok}[1]{\textcolor[rgb]{0.56,0.35,0.01}{\textbf{\textit{#1}}}}
\usepackage{longtable,booktabs,array}
\usepackage{calc} % for calculating minipage widths
% Correct order of tables after \paragraph or \subparagraph
\usepackage{etoolbox}
\makeatletter
\patchcmd\longtable{\par}{\if@noskipsec\mbox{}\fi\par}{}{}
\makeatother
% Allow footnotes in longtable head/foot
\IfFileExists{footnotehyper.sty}{\usepackage{footnotehyper}}{\usepackage{footnote}}
\makesavenoteenv{longtable}
\usepackage{graphicx}
\makeatletter
\def\maxwidth{\ifdim\Gin@nat@width>\linewidth\linewidth\else\Gin@nat@width\fi}
\def\maxheight{\ifdim\Gin@nat@height>\textheight\textheight\else\Gin@nat@height\fi}
\makeatother
% Scale images if necessary, so that they will not overflow the page
% margins by default, and it is still possible to overwrite the defaults
% using explicit options in \includegraphics[width, height, ...]{}
\setkeys{Gin}{width=\maxwidth,height=\maxheight,keepaspectratio}
% Set default figure placement to htbp
\makeatletter
\def\fps@figure{htbp}
\makeatother
\setlength{\emergencystretch}{3em} % prevent overfull lines
\providecommand{\tightlist}{%
  \setlength{\itemsep}{0pt}\setlength{\parskip}{0pt}}
\setcounter{secnumdepth}{5}
\usepackage{booktabs}
\ifLuaTeX
  \usepackage{selnolig}  % disable illegal ligatures
\fi
\usepackage[]{natbib}
\bibliographystyle{plainnat}
\usepackage{bookmark}
\IfFileExists{xurl.sty}{\usepackage{xurl}}{} % add URL line breaks if available
\urlstyle{same}
\hypersetup{
  pdftitle={CCHS Example},
  pdfauthor={Asal Aslemand, Andrew Gad, Jaiditya Dev, and Yihan Wang,},
  hidelinks,
  pdfcreator={LaTeX via pandoc}}

\title{CCHS Example}
\author{Asal Aslemand, Andrew Gad, Jaiditya Dev, and Yihan Wang,}
\date{2024-10-01}

\begin{document}
\maketitle

{
\setcounter{tocdepth}{1}
\tableofcontents
}
\chapter{About This Book}\label{about-this-book}

This book presents a comprehensive analysis of the Canadian Community Health Survey, focusing on mental health perceptions among youths. The survey's rich dataset allows us to explore various aspects influencing mental health, including social, economic, and psychological factors.

\section{Purpose of the Study}\label{purpose-of-the-study}

The main goal of this study is to identify factors that are strongly associated with positive mental health outcomes among Canadian youths. Insights from this analysis could inform policy makers and healthcare providers about effective strategies to improve mental health services.

\section{How to Use This Book}\label{how-to-use-this-book}

Each chapter of this book corresponds to a distinct aspect of the analysis:
- \textbf{Chapter 1: Introduction and Background Information} - Provides an overview of the survey methodology and the importance of studying mental health.
- \textbf{Chapter 2: Data and Methodology} - Details the statistical methods used for the analysis, including logistic regression and categorical data analysis.
- \textbf{Chapter 3: Results and Discussion} - Discusses the findings from the analysis, supported by visualizations and tables.
- \textbf{Chapter 4: Conclusions} - Summarizes the key takeaways and suggests future research directions.

\section{Render This Book}\label{render-this-book}

You can render this book into HTML or PDF formats directly from the RStudio IDE or from the R console using the following command:

\begin{Shaded}
\begin{Highlighting}[]
\NormalTok{bookdown}\SpecialCharTok{::}\FunctionTok{render\_book}\NormalTok{(}\StringTok{\textquotesingle{}index.Rmd\textquotesingle{}}\NormalTok{, }\StringTok{\textquotesingle{}bookdown::gitbook\textquotesingle{}}\NormalTok{)}
\end{Highlighting}
\end{Shaded}

\chapter{Introduction and Background Information}\label{introduction-and-background-information}

Welcome to our analysis of the Canadian Community Health Survey. This book aims to explore various health outcomes by examining data collected across different demographics and regions.

\section{Survey Overview}\label{survey-overview}

The Canadian Community Health Survey (CCHS) is an annual survey that gathers health-related data from individuals aged 12 and older. The survey covers various aspects, including health status, healthcare utilization, and social determinants of health.

\section{Importance of the Study}\label{importance-of-the-study}

Understanding the factors that influence health outcomes is crucial for developing effective public health policies and interventions. This study provides insights into the patterns and trends in health behaviors among Canadians.

\subsection*{Methodology Overview}\label{methodology-overview}
\addcontentsline{toc}{subsection}{Methodology Overview}

This section provides a brief overview of the methodologies used in our analysis, including the statistical techniques and the rationale behind choosing specific models.

\begin{itemize}
\tightlist
\item
  \textbf{Data Collection}: How the data was collected and any limitations.
\item
  \textbf{Statistical Methods}: Overview of the methods used for analyzing the data.
\item
  \textbf{Study Goals}: The main objectives that guide the analysis.
\end{itemize}

\subsection*{Key Definitions}\label{key-definitions}
\addcontentsline{toc}{subsection}{Key Definitions}

\begin{itemize}
\tightlist
\item
  \textbf{Health Indicator}: A measure that reflects, or indicates, the state of health of persons in a defined population.
\item
  \textbf{Demographic Factors}: Variables such as age, race, income level, etc., that are used to categorize survey respondents.
\end{itemize}

\section{Organization of This Book}\label{organization-of-this-book}

This book is divided into several chapters, each addressing different aspects of the survey findings:

\begin{itemize}
\tightlist
\item
  \textbf{Chapter 2: Data and Methodology} - Details the statistical methods used for the analysis.
\item
  \textbf{Chapter 3: Results and Discussion} - Discusses the findings from the analysis, supported by visualizations and tables.
\item
  \textbf{Chapter 4: Conclusions and Recommendations} - Summarizes the key takeaways and suggests future research directions.
\end{itemize}

\chapter{Data and Methodology}\label{data-and-methodology}

This chapter outlines the methodologies used and the data sources employed in our analysis of the Canadian Community Health Survey.

\section{Data Sources}\label{data-sources}

We utilized the 2022 release of the Canadian Community Health Survey, which covers a wide demographic range and includes numerous health-related variables.

\begin{itemize}
\tightlist
\item
  \textbf{Population Covered}: The survey includes responses from over 30,000 Canadians aged 12 and above.
\item
  \textbf{Sampling Method}: Description of the stratified sampling technique used to ensure a representative sample of the Canadian population.
\end{itemize}

\section{Analytical Techniques}\label{analytical-techniques}

We applied several statistical methods to analyze the data, focusing on trends and correlations that impact health outcomes.

\begin{itemize}
\tightlist
\item
  \textbf{Descriptive Statistics}: Initial analysis to understand the distribution and basic features of the data.
\item
  \textbf{Regression Analysis}: Employed to determine the relationships between various health indicators and demographic factors.
\item
  \textbf{Machine Learning Models}: Utilized for predicting health outcomes based on input variables.
\end{itemize}

\subsection*{Data Cleaning Procedures}\label{data-cleaning-procedures}
\addcontentsline{toc}{subsection}{Data Cleaning Procedures}

Before analysis, we conducted thorough data cleaning steps:

\begin{itemize}
\tightlist
\item
  \textbf{Handling Missing Data}: Techniques used to address missing values in the dataset.
\item
  \textbf{Outlier Detection}: Methods for identifying and handling outliers in the data.
\end{itemize}

\subsection*{Ethical Considerations}\label{ethical-considerations}
\addcontentsline{toc}{subsection}{Ethical Considerations}

\begin{itemize}
\tightlist
\item
  \textbf{Data Privacy}: Measures taken to ensure the confidentiality and anonymity of survey respondents.
\end{itemize}

\chapter{Results and Discussion}\label{results-and-discussion}

This chapter presents the findings from our analysis of the health survey data and discusses their implications.

\section{Key Findings}\label{key-findings}

\begin{itemize}
\tightlist
\item
  \textbf{Health Trends}: Overview of significant health trends identified in the survey data, such as changes in physical activity levels and dietary habits over the years.
\item
  \textbf{Demographic Differences}: How health outcomes vary across different demographic groups.
\end{itemize}

\subsection*{Detailed Analysis Results}\label{detailed-analysis-results}
\addcontentsline{toc}{subsection}{Detailed Analysis Results}

\begin{itemize}
\tightlist
\item
  \textbf{Statistical Significance}: Presentation of the statistical tests used and the significance of the findings.
\item
  \textbf{Model Performance}: Evaluation of the accuracy and reliability of the machine learning models employed.
\end{itemize}

\section{Discussion}\label{discussion}

\begin{itemize}
\tightlist
\item
  \textbf{Interpretation of Results}: What the results suggest about the health of Canadians.
\item
  \textbf{Comparison with Previous Studies}: How these findings align or contrast with previous research.
\end{itemize}

\subsection*{Limitations of the Study}\label{limitations-of-the-study}
\addcontentsline{toc}{subsection}{Limitations of the Study}

\begin{itemize}
\tightlist
\item
  \textbf{Scope of Data}: Limitations due to the scope of the survey data.
\item
  \textbf{Methodological Limitations}: Constraints encountered in the analytical techniques used.
\end{itemize}

  \bibliography{book.bib,packages.bib}

\end{document}
